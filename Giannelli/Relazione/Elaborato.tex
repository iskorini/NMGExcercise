\documentclass[a4paper, 10pt]{article}
\usepackage{amsmath}
\usepackage{amsthm}
\usepackage{amsfonts}
\usepackage{amssymb}
\usepackage[italian]{babel}
\usepackage[utf8x]{inputenc}
\usepackage{mathtools}
\usepackage{epigraph}
\usepackage{listings}
\usepackage{amsfonts, mathtools, placeins, floatflt, xfrac, multirow, indentfirst, fancyhdr,
            amssymb, latexsym, amsthm, eucal, float, tabularx, mdframed, bigdelim, blkarray,
            relsize, lettrine, xcolor, tikz, caption, geometry, perpage}
  \usepackage{courier}

\usepackage[tiny]{titlesec}
\usepackage{lmodern}
\usepackage{tikz}
\usepackage[object=vectorian]{pgfornament}
\usetikzlibrary{calc}

\newcommand{\sectionline}[2]{%
  \nointerlineskip \vspace{.5\baselineskip}\hspace{\fill}
  {\color{#1}
    \resizebox{0.5\linewidth}{2ex}
    {{%
    {\begin{tikzpicture}
    \node  (C) at (0,0) {};
    \node (D) at (9,0) {};
    \path (C) to [ornament=#2] (D);
    \end{tikzpicture}}}}}%
    \hspace{\fill}
    \par\nointerlineskip \vspace{.5\baselineskip}
  }



 \makeatletter

\def\vhrulefill#1{\leavevmode\leaders\hrule\@height#1\hfill \kern\z@}

\makeatother

\newcommand{\persendchap}{

	\center 	{\vhrulefill{0.9pt} \pgfornament[scale = 0.50, symmetry=v, ydelta = -0.5pt]{11} FINE CAPITOLO \pgfornament[scale = 0.50, ydelta=-0.5pt]{11} \vhrulefill{0.9pt} }
	\flushleft

}

%\renewcommand\thechapter{\arabic{chapter}}
\definecolor{mygreen}{RGB}{28,172,0} 
\definecolor{mylilas}{RGB}{170,55,241}

\lstloadlanguages{Matlab, Fortran}
 \lstset{
         basicstyle=\footnotesize\ttfamily, 
         numbers=left,              
         morekeywords={matlab2tikz},
         numberstyle=\tiny,        
         numbersep=5pt,           
         tabsize=2,                 
         extendedchars=true,        
         breaklines=true,           
         keywordstyle=\color{blue},
            frame=b,
         showspaces=false,          
         showtabs=false,            
         xleftmargin=17pt,
         framexleftmargin=17pt,
         framexrightmargin=5pt,
         framexbottommargin=4pt,
         showstringspaces=false     
              morekeywords=[2]{1}, keywordstyle=[2]{\color{black}},
    identifierstyle=\color{black},%
    stringstyle=\color{mylilas},
    commentstyle=\color{mygreen},%
    showstringspaces=false,
 }


\usepackage{caption}
\DeclareCaptionFont{white}{\color{white}}
\DeclareCaptionFormat{listing}{\colorbox[cmyk]{0.43, 0.35, 0.35,0.01}{\parbox{\textwidth}{\hspace{15pt}#1#2#3}}}
\captionsetup[lstlisting]{format=listing,labelfont=white,textfont=white, singlelinecheck=false, margin=0pt, font={bf,footnotesize}}
\newtheorem{mydef}{Definizione}
\begin{document}
\renewcommand{\lstlistingname}{Codice}

\begin{titlepage}
	\begin{center}
 		\includegraphics[scale=0.30]{logo/LOGO}\\
				
		\vspace{1.0cm}
		\Huge  Relazione \\ \vspace{0.3cm} di\\ \vspace{0.3cm} Metodi Numerici per la Grafica \\
		\vspace{1.5 cm}
		\Large  Di \vspace{0.2cm} \\ Federico Schipani 

		 \vfill                   
      A.A. 2017-2018
      \vspace{1cm}

    %  \begin{tikzpicture}
    %\node  (C) at (0,0) {};
    %\node (D) at (9,0) {};
    %\path (C) to [ornament=70] (D);
    %\end{tikzpicture}
      \vfill   
	\end{center}
\end{titlepage}

\tableofcontents
\newpage
%\vspace{30mm}

%\begin{tikzpicture}[every node/.style={inner sep=0pt}]
%\node[text width=8cm,align=center](Text){%
%
%
%La matematica è una delle manifestazioni più significative dell'amore per la sapienza. Come tale è caratterizzata da un lato da una grande libertà, 
%dall'altro dall'intuizione che il mondo è fatto di cose visibili e invisibili e la matematica ha forse una capacità, unica fra le altre scienze, 
%di passare dall'osservazione delle cose visibili all'immaginazione delle cose invisibili. Questo forse è il segreto della forza della matematica.
%
%
%\bigskip
%
%\vspace{24pt}
%  Ennio De Giorgi
%} ;
%\node[shift={(-1cm,1cm)},anchor=north west](CNW)  at (Text.north west)
%               {\pgfornament[width=2cm]{61}};
%\node[shift={(1cm,1cm)},anchor=north east](CNE)   at (Text.north east)
%               {\pgfornament[width=2cm,symmetry=v]{61}};
%\node[shift={(-1cm,-1cm)},anchor=south west](CSW) at (Text.south west)
%               {\pgfornament[width=2cm,symmetry=h]{61}};
%\node[shift={(1cm,-1cm)},anchor=south east](CSE)  at (Text.south east)
%               {\pgfornament[width=2cm,symmetry=c]{61}};
%\pgfornamenthline{CNW}{CNE}{north}{87}
%\pgfornamenthline{CSW}{CSE}{south}{87}
%\pgfornamentvline{CNW}{CSW}{west}{87}
%\pgfornamentvline{CNE}{CSE}{east}{87}
%\end{tikzpicture}
%
%
%\titleformat{\chapter}[display]
%  {\normalfont\huge\bfseries\raggedleft}
%  {\begin{tikzpicture}
%  \node[text width=3cm,align=center] (chapnum)
%    {\fontsize{100}{130}\color{gray}\selectfont\thechapter};%
%  \node[shift={(-1cm,1cm)},anchor=north west](CNW)
%    at (chapnum.north west) {\pgfornament[width=1.75cm]{61}};
%  \node[shift={(1cm,1cm)},anchor=north east](CNE)
%    at (chapnum.north east) {\pgfornament[width=1.75cm,symmetry=v]{61}};
%  \node[shift={(-1cm,-1cm)},anchor=south west](CSW)
%    at (chapnum.south west) {\pgfornament[width=1.75cm,symmetry=h]{61}};
%  \node[shift={(1cm,-1cm)},anchor=south east](CSE)
%    at (chapnum.south east) {\pgfornament[width=1.75cm,symmetry=c]{61}};
%  \end{tikzpicture}
%  }
%  {20pt}
%  {\Huge}

\section{La Base delle B-Spline}

Dato un vettore esteso dei nodi
$$ \mathbf{t} =  \left\{ \underbrace{t_{0}, \dots, t_{k-2}}_{k-1}, \underbrace{t_{k-1}, \dots, t_{n+1}}_{\tau_0, \tau_1, \dots, \tau_L}, \underbrace{t_{n+2}, \dots, t_{n+k}}_{k-1} \right\} $$
con
$$\mathbf{t_0} \leq t_1 \leq \dots t_{k+1} < t_k \dots < t_{n+1} \leq t_{n+2} \leq \dots \leq t_{n+k}$$
possiamo definire la base delle B-Spline su nodi semplici tramite la relazione ricorrente di \textit{Cox-De Boor}.
\begin{mydef}
  Le B-Spline di ordine $1$, oppure grado $0$ sono definite come:
$$N_{i, 1}(t) = \begin{cases} 1, & \text{se } t\in[t_i, t_{i+1}] i = 0, \dots, n+k-1 \\ 0, & \text{altrimenti} \end{cases}$$
  Altrimenti le B-Spline di ordine $r \leq k$ sono definite ricorsivamente, per $r > 1$, come:
  $$N_{i, r}(t) = \omega_{i,r}(t)N_{i, r-1}(t) + [1-\omega_{i+1, r}(t)]N_{i+1, r-1}$$
  dove
$$\omega_{i,r}(t) = \begin{cases} \frac{t-t_i}{t_{i+r-1}-t_i}, & \text{se } t<t_{i+r-1} \\ 0, & \text{altrimenti} \end{cases}$$
\end{mydef}
Le B-Spline possono anche essere definite su una partizione nodale la cui molteplicità $m_i$ di un generico nodo $\tau_i$ è più alta di $1$,
quindi su nodi multipli. In questo caso il vettore esteso dei nodi diventa:
$$ \mathbf{t} =  \left\{ \underbrace{t_{0}, \dots, t_{k-2}}_{k-1}, \underbrace{t_{k-1}, \dots, t_{n+1}}_{\tau_0, \tau_1, \dots, \tau_1 \dots, \tau_L}, \underbrace{t_{n+2}, \dots, t_{n+k}}_{k-1} \right\} $$
con $\tau_i$ ripetuto a seconda della sua molteplicità $m_i$ con $i = 1, \dots, L-1$ in $\mathbf{t}$, e
$$\mathbf{t_0} \leq t_1 \leq \dots t_{k+1} \leq  t_k \dots \leq  t_{n+1} \leq t_{n+2} \leq \dots \leq t_{n+k}$$.
La definizione della base delle B-Spline di \textit{Cox-De Boor} non cambia, ma bisogna stare attenti in quanto 
$\omega_{i,r}(t)$ può diventare nullo per qualche valore $r$  a causa dei nodi multipli.

\end{document}
